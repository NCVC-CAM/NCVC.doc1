%!TEX root = ../NCVC.tex

\mysection{基本編}

\subsection{CADでの作図}

\begin{minipage}[t]{0.4\textwidth}
 まずは基本的な加工を行うための基本的な作図方法を解説します.
図\ref{fig:sample1.jww} のような図形を書きましょう.
切削対象(ワーク)を示す矩形と,その矩形左下に円を1つ.
「NCVC」という文字は,線をつなぎ合わせたデータです.
\end{minipage}
\begin{minipage}[t]{0.6\textwidth}
\vspace*{-2zh}
\begin{figure}[H]
\centering
\includegraphics[scale=0.8]{No1/fig/sample1.pdf}
\caption{サンプル図形}
\label{fig:sample1.jww}
\end{figure}
\end{minipage}

\vspace*{2zh}
 NCVCはCADでの作図情報を全て読み込むのではなく,
特定のレイヤ情報を元に作図データを読み込みます.
CADでの作図において必要とされる補助線や寸法線等が加工データには必要なく,
これらを選別するための仕様です.

 その選別方法は『必要なレイヤに名前を付ける』こと.
図\ref{fig:sampleLayer.png} は図\ref{fig:sample1.jww} のレイヤ情報ですが,
0番レイヤに「ORIGIN」という名前,
1番レイヤに「CAM\_LINE」という名前を付けています.
それぞれ機械原点と切削軌跡を示し,この2つのレイヤは必須です
\footnote{実は機械原点レイヤは必須ではありません.詳細は【穴加工】の節で解説しています.}.
機械原点レイヤには工作機械のXY原点を示す円を1つだけ作図.
大きさは任意ですが,円の中心がXYの原点となります.
切削軌跡 CAM\_LINE レイヤには刃物のパス,
すなわち削りたい図形を書きます.
他,ワーク矩形を示す補助線等は別のレイヤに書きます.
レイヤに名前を付ける方法は,それぞれのCAD操作に準拠して下さい.
なお,全てのデータにおいて線種,線色は関係ありません.

\vspace*{1zh}
\begin{figure}[H]
\centering
\includegraphics{No1/fig/sampleLayer.png}
\caption{レイヤ一覧}
\label{fig:sampleLayer.png}
\end{figure}

 作図が終わればCADデータをDXF形式で保存します
\footnote{
    Jw\_cadの場合,DXF形式で保存する必要はありません.NCVCはJWW形式を直接読み込むことが可能です.
    詳細は【パワーユーザ編】の【アドイン作成のすすめ】を参照して下さい.
}.
NCVCにCADデータを読み込ませるためDXF形式で保存する必要がありますが,
多くの場合,DXF形式で保存するとそのCAD独自のデータが失われるため,
使用しているCAD独自の形式でも保存しておきましょう.

\subsection{CADデータの読み込み}

\begin{minipage}[t]{0.5\textwidth}
 NCVCでDXF形式のCADデータを読み込みます.
が,その前に確認.
NCVCの \menu{オプション>DXF関連の設定} をクリックし,
NCVCが読み込むレイヤ名を設定して下さい.
デフォルトで先ほど設定した値になっていると思います.
基本編では[従来互換]のみ解説しますので,図\ref{fig:NCVCsetup.png} の通り設定して下さい.
この値は任意です.CAD側の設定と合わせて下さい.
無事読み込めると原点を示す十字(大きさは原点円の直径)と切削対象のパスが表示されます.
原点レイヤと切削レイヤ以外に作図した情報,
例えば,図\ref{fig:sampleLayer.png} の4番レイヤに書いたワークを表す矩形は読み込まれません(図\ref{fig:NCVCread.png}).
CADでの線種・線色は無視され,NCVCの設定に基づき表示されます.
詳細はリファレンスの表示属性を参照してください.
\end{minipage}
\begin{minipage}[t]{0.5\textwidth}
\vspace*{-2zh}
\begin{figure}[H]
\centering
\includegraphics[scale=0.8]{No1/fig/NCVCsetup.png}
\caption{読み込みレイヤ設定}
\label{fig:NCVCsetup.png}
\end{figure}
\end{minipage}

\begin{figure}[H]
\centering
\includegraphics[scale=0.4]{No1/fig/NCVCread.png}
\caption{CADデータの読み込み}
\label{fig:NCVCread.png}
\end{figure}

\subsection{加工条件の設定}

 いよいよCADデータからGコードを生成するわけですが,ご覧の通り読み込んだCADデータは2次元です.
工作機械のZ軸方向の移動はどうやって制御するのでしょうか?
答えは[加工条件]の中にあります.
\menu{オプション>切削パラメータの設定} をクリックし,条件ファイル(nciファイル)を選択します.
標準で用意されている[Init.nci]の設定を変更しましょう.

 条件ファイルを選択すると図\ref{fig:init.nci.png} のダイアログが表示されます.
ここで重要なのが切削原点(G92)のZ値とR点,切り込みパラメータの3つです.

 図\ref{fig:XZ-crop.pdf} は工作機械を正面から見た図,上下にZ軸,左右にX軸です.
ワークをセットしたあと,ワーク平面を基準にZセンサー等でZ軸の位置決めを行います.
これを切削原点(G92)のZ値とします.
Zセンサーの厚みが100mmなら100と入力です.
センサーでの調整後,好みの位置に移動させてもかまいません.
無論そのときは移動した座標値を入力して下さい.

 次に切り込みですが,イメージ通り.ワークに何ミリ切り込むかという設定です.
最後にR点ですが,これは次の切削領域,この例で言うと``\,N\,''を削って``\,C\,''に移動するときのZ値を指定します.
Z軸の初期位置(原点)で移動してもかまわないのですが,初期位置は高く設定する傾向があるため,効率よく移動できる下限値と考えて下さい.
この設定ではワーク平面上空1mmの所で刃物が次の切削領域へ高速移動します.

 ワーク平面を基準に値を選びましたが,Zセンサー調整後の位置を基準,
すなわち,ワーク上空10mmの位置をZ軸の原点(G92Zをゼロ)としたとき,この例ではR点が-9mm,切り込みは-12mmとなります.
意味は同じですから各自の好みや考えやすい方で指示して下さい.

 他,主軸回転数や送り速度など,ワーク材質に合わせて設定します.

\begin{minipage}{0.5\textwidth}
\begin{figure}[H]
\centering
\includegraphics[scale=0.7]{No1/fig/init.nci.png}
\caption{加工条件の設定}
\label{fig:init.nci.png}
\end{figure}
\end{minipage}
\begin{minipage}{0.5\textwidth}
\begin{figure}[H]
\centering
\includegraphics[scale=0.8]{No1/fig/XZ-crop.pdf}
\caption{Z軸における各パラメータの関係}
\label{fig:XZ-crop.pdf}
\end{figure}
\end{minipage}

\subsection{Gコードの生成}

\begin{minipage}[t]{0.5\textwidth}
 加工条件の設定ができればあとはNCVCの仕事.
\menu{ファイル>NCデータへの変換>単一条件(従来互換)} をクリック.
出力ファイル名(自動設定)と条件ファイルを指定(図\ref{fig:make.png})し,OKをクリックすれば
\end{minipage}
\begin{minipage}[t]{0.5\textwidth}
\vspace*{-2zh}
\begin{figure}[H]
\centering
\includegraphics[scale=0.7]{No1/fig/make.png}
\caption{Gコードの出力と条件ファイルの指示}
\label{fig:make.png}
\end{figure}
\end{minipage}

\vspace*{2zh}
 おめでとうございます!見事Gコードが生成できました.
図\ref{fig:make.png} で[NC生成後に開く]にチェックが入っていると,即座に結果を確認することが出来ます.
図\ref{fig:sim.png} にGコードのシミュレーション結果を示します.

\begin{figure}[H]
\centering
\includegraphics[scale=0.4]{No1/fig/sim.png}
\caption{Gコードシミュレーション画面}
\label{fig:sim.png}
\end{figure}

\vspace*{2zh}
\begin{itembox}[l]{ここまでの【まとめ】}
(1) CADでの操作
\begin{itemize}
\item 工作機械のXY原点を示す円を原点レイヤに作図
\item 刃物の軌跡を切削レイヤに作図
\item 原点レイヤと切削レイヤに名前を付ける
\item 線種・線色は無視され,NCVCの表示属性により表示される
\end{itemize}
(2) NCVCでの操作
\begin{itemize}
\item CADデータを読み込むために,読み込みレイヤの設定を行う
\item Z軸の原点や切り込み量は加工条件で設定する
\end{itemize}
\end{itembox}
