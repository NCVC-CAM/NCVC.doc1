%!TEX root = ../NCVC.tex

\mysection{リファレンス}

\subsection{メニュー}
 NCVCは,開いたデータタイプによってメニューが切り替わります.
何も開いていない初期メニュー,CAD系,NC系に分けて記載しました.
それぞれのステージごとに参照して下さい.

\subsubsection{初期(共通)メニュー}
1) ファイル\\ \vspace*{-2zh}
\begin{table}[H]
\begin{tabular}{|p{5cm}Il}
\cline{1-1}
開く & CADデータ,NCデータ(Gコード)を開きます.\\ \cline{1-1}
アドイン & ファイル系アドインがあれば表示.以下同じ \\ \cline{1-1}
(読み込み履歴)& 読み込み履歴が10個まで表示 \\ \cline{1-1}
アプリケーションの終了 & NCVCの終了 \\ \wcline{1-1}
\end{tabular}
\end{table}

2) 表示\\ \vspace*{-2zh}
\begin{table}[H]
\begin{tabular}{|p{5cm}Il}
\cline{1-1}
ツールバー & ツールバーのカスタマイズが行えます(p.\pageref{sec:toolbar}) \\
ステータスバー & ステータスバーの表示,非表示 \\ \wcline{1-1}
\end{tabular}
\end{table}

3) オプション\\ \vspace*{-2zh}
\begin{table}[H]
\begin{tabular}{|p{5cm}Il}
\cline{1-1}
工作機械の設定 & 工作機械情報の設定(p.\pageref{fig:kikai.png}, \pageref{sec:kikai}) \\
工作機械情報のテキスト編集 & 工作機械の設定をダイアログではなくエディタにて行う \\ \cline{1-1}
DXF関連の設定 & CADデータの読み込みレイヤ設定など(p.\pageref{fig:ReadSetup.png}, \pageref{fig:ReadSetup2.png}, \pageref{fig:ReadSetup3.png}, \pageref{fig:25d-setup.png}, \pageref{sec:ReadSetup})\\ \cline{1-1}
切削パラメータの設定 & 加工条件設定(p.\pageref{fig:init.nci.png}, \pageref{fig:hole.png}, \pageref{fig:move-setup.png}, \pageref{fig:deep-setup.png}, \pageref{sec:init.nci}) \\
切削パラメータのテキスト編集 & 加工条件設定をエディタにて行う \\ \cline{1-1}
表示属性 & 画面色などの設定.属性のエクスポートやインポートも可(p.\pageref{sec:gamen}) \\ \cline{1-1}
外部アプリケーションの設定 & NCVCから呼び出す外部アプリの設定(p.\pageref{sec:app}) \\
拡張子の設定 & NCVCが開く拡張子の設定(p.\pageref{sec:ext}) \\ \wcline{1-1}
\end{tabular}
\end{table}

4) 外部アプリ\\
 [外部アプリケーションの設定](p.\pageref{sec:app})で登録した順に表示される.
上記[~のテキスト編集]は外部アプリケーションの1番目に渡されるため,1番目にはテキストエディタを登録した方が良い.\\

5) ヘルプ\\ \vspace*{-2zh}
\begin{table}[H]
\begin{tabular}{|p{5cm}Il}
\cline{1-1}
目次 & 古いバージョンのヘルプが表示されます \\ \cline{1-1}
NCVCのバージョン情報 & \\
アドインについて & アドインがあれば簡易ヘルプが表示されます \\ \wcline{1-1}
\end{tabular}
\end{table}

\newpage
\subsubsection{CADデータ系}
1) ファイル\\ \vspace*{-2zh}
\begin{table}[H]
\begin{tabular}{|p{5cm}Il}
\cline{1-1}
開く & \\
閉じて開く & \\
閉じる & \\ \cline{1-1}
加工情報の保存 & CADデータをNCVC形式で保存 \\
加工情報を名前を付けて保存 &  本バージョンではデータ交換が主な目的 \\ \cline{1-1}
アドイン & \\ \cline{1-1}
NCデータへの変換 & 加工データの生成(p.\pageref{fig:make.png}, \pageref{fig:25d-make1.png}, \pageref{fig:25d-make3.png}, \pageref{fig:direction-setup}) \\ \cline{1-1}
(読み込み履歴)&  \\ \cline{1-1}
アプリケーションの終了 & \\ \wcline{1-1}
\end{tabular}
\end{table}

2) 編集\\ \vspace*{-2zh}
\begin{table}[H]
\begin{tabular}{|p{5cm}Il}
\cline{1-1}
切り取り & 使用できません \\
コピー & CADデータの描画イメージをクリップボードにコピー \\
貼り付け & 使用できません \\ \cline{1-1}
原点調整 & 読み込み後,数値オフセット・矩形位置で原点を移動(p.\pageref{sec:origin}) \\ \wcline{1-1}
\end{tabular}
\end{table}

3) 表示\\ \vspace*{-2zh}
\begin{table}[H]
\begin{tabular}{|p{5cm}Il}
\cline{1-1}
ツールバー & \\
ステータスバー & \\ \cline{1-1}
レイヤ & 読み込んだレイヤ情報(p.\pageref{fig:25d-read.png}, \pageref{fig:direction.png}, \pageref{sec:layer})\\ \cline{1-1}
図形フィット & 現在のウィンドウに収まるように図形を表示 \\
拡大 & \\
縮小 & \\
直前の拡大率 & \\
上移動 & \\
下移動 & \\
左移動 & \\
右移動 & \\ \wcline{1-1}
\end{tabular}
\end{table}

4) ウィンドウ\\ \vspace*{-2zh}
\begin{table}[H]
\begin{tabular}{|p{5cm}Il}
\cline{1-1}
重ねて表示 & \\ 
並べて表示 & \\
アイコンの整列 & \\ \cline{1-1}
次のウィンドウ & \\ \cline{1-1}
閉じる & \\
全て閉じる & \\ \cline{1-1}
(開いているウィンドウ一覧)& \\ \wcline{1-1}
\end{tabular}
\end{table}

5) オプション\\
6) 外部アプリ\\
7) ヘルプ\\
 初期メニューと同じ

\newpage
\subsubsection{NCデータ系}
1) ファイル\\ \vspace*{-2zh}
\begin{table}[H]
\begin{tabular}{|p{5cm}Il}
\cline{1-1}
開く & \\
カーソル位置に読み込み & Gコードリストがアクティブのとき有効 \\
閉じて開く & \\
閉じる & \\ \cline{1-1}
上書き保存 & \\
名前を付けて保存 & \\
DXF出力 & シミュレーション結果を各平面別にDXF出力(p.\pageref{sec:output-dxf}) \\ \cline{1-1}
アドイン & \\ \cline{1-1}
(読み込み履歴)&  \\ \cline{1-1}
アプリケーションの終了 & \\ \wcline{1-1}
\end{tabular}
\end{table}

2) 編集\\ \vspace*{-2zh}
\begin{table}[H]
\begin{tabular}{|p{5cm}Il}
\cline{1-1}
切り取り & 使用できません \\
コピー & 現在アクティブなペイン情報をクリップボードにコピー \\
貼り付け & 使用できません \\ \wcline{1-1}
\end{tabular}
\end{table}

3) 表示\\ \vspace*{-2zh}
\begin{table}[H]
\begin{tabular}{|p{5cm}Il}
\cline{1-1}
ツールバー & \\
ステータスバー & \\ \cline{1-1}
図形フィット & \\
全てのペインを図形フィット & 4面表示のみ有効 \\
拡大 & \\
縮小 & \\
直前の拡大率 & \\
上移動 & \\
下移動 & \\
左移動 & \\
右移動 & \\ \cline{1-1}
補助矩形 & 最大加工矩形・ワーク矩形の表示(p.\pageref{sec:maxrect}) \\ \cline{1-1}
トレース & \\
指定行 & Gコードリストの指定行に移動(p.\pageref{sec:jump})\\ \wcline{1-1}
\end{tabular}
\end{table}

4) ウィンドウ\\
 CADデータ系と同じ\\

5) オプション\\
6) 外部アプリ\\
7) ヘルプ\\
 初期メニューと同じ