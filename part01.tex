%!TEX root = ./NCVC.tex

\section{はじめに}
 NCVC(NC Viewer and Converter)はCAMソフトです.
主にCAD情報からNC工作機械を動かすためのGコードを生成するアプリケーションです.
NCVCにCAD(作図)機能は付いていません.
作図は別途CADソフトで行って下さい.
CADソフトは Jw\_cad for Windows (以降 Jw\_cad と略記)を推奨しますが,
以下の条件に当てはまるCADならNCVCの入力源としてそのまま使えます.

\begin{itemize}
    \item R12形式のDXFが出力可能なもの
    \item DXFにレイヤ情報(レイヤ名)が出力できるもの
\end{itemize}

 ほとんどのCADが当てはまると思います.
普段使い慣れたCADソフトをご使用下さい.
逆に言うとNCVCを使うためにわざわざ新たなCADの操作方法を覚える必要が無いということです.
以降本解説書でのCAD操作は Jw\_cad をベースに解説します.

 残念ながらCADを使ったことが無いという方,本解説書ではCADの使用を前提にしています.
上記条件を満たしていればドロー系ソフトでもかまいませんが,正確な作図が要求されます.
まずはCADでの作図方法を習得して下さい.

 もう1つ,NCVCにはGコードのシミュレーション機能がありますが,
冒頭で述べたとおりNCVCはGコードの生成を主な目的としています.
全てのGコードには対応していませんので,サポートされるGコードは付録の対応Gコード一覧を参照して下さい.
また,シミュレーション結果と工作機械の動作が必ず一致するとも限りません.
実際の加工にはくれぐれもご注意下さい.
