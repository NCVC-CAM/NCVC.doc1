%!TEX root = ../NCVC.tex

\mysection{中級編}

\subsection{移動レイヤ}
\label{sec:move}
 NCVCには,原点レイヤ・切削レイヤの他にあと3つのレイヤ情報を読み込む機能があります.
ここではそのうちの2つ,加工開始位置指示レイヤと強制移動指示レイヤを解説します.

\subsubsection{加工開始位置指示レイヤ}
 図\ref{fig:sample3.pdf} のような加工を考えます.
原点はワーク矩形左下で円を内側から螺旋状に切削したいのですが,これだけでは思ったようなGコードを生成できません.

 NCVCは次の切削データを検索するとき,現在位置に最も近い座標を検索します.
したがって,原点から一番近い座標である外側の座標からGコードの生成を始めます.

 これを回避するため,NCVCでは[加工開始位置指示レイヤ]を用意しています.
CADでの作図で原点・切削の両レイヤとは別のレイヤを用意し,円を1つ作図して下さい.
レイヤ名も設定です.作図方法は原点指示と同じ,円の中心が加工開始座標となります.

 NCVCでの設定は【\ref{sec:ReadCAD} CADデータの読み込み】と同じです.
[読み込みレイヤ2]のタブをクリックし,NCVCが読み込むレイヤ名を設定して下さい(図\ref{fig:ReadSetup3.png}).
[読み込みレイヤ2]の設定は必須ではありませんが,CAD側で意図的に作図しない限りNCVCは読み込みませんので,常にこの設定にしても問題ありません.
レイヤ名は,原点・切削の両レイヤと同様に任意です.CAD側の設定と合わせて下さい.
加工開始指示を指示したシミュレーション結果を図\ref{fig:sample3.png} に示します.

\begin{minipage}{0.5\textwidth}
\begin{figure}[H]
\centering
\includegraphics{No3/fig/start-crop.pdf}
\caption{サンプル図形}
\label{fig:sample3.pdf}
\end{figure}
\end{minipage}
\begin{minipage}{0.5\textwidth}
\begin{figure}[H]
\centering
\includegraphics[scale=0.7]{No3/fig/ReadSetup3.png}
\caption{読み込みレイヤ設定}
\label{fig:ReadSetup3.png}
\end{figure}
\end{minipage}

\begin{figure}[H]
\centering
\includegraphics[scale=0.55]{No3/fig/sample3.png}
\caption{加工開始位置を追加したGコードシミュレーション画面}
\label{fig:sample3.png}
\end{figure}

 加工開始位置指示レイヤにはもう1つ機能があります.
例えば図\ref{fig:approach-img.png} のようなワーク形状の場合,原点からの最短移動ではワークに干渉してしまいます.
この場合,図\ref{fig:approach.pdf} のように干渉しないような移動軌跡を加工開始位置指示レイヤに作図することで,原点からの移動動作を制御することができます.
Z軸の初期座標を上げることで回避できる場合もあります.用途に応じてご使用下さい.

\begin{minipage}{0.5\textwidth}
\begin{figure}[H]
\centering
\includegraphics[width=\textwidth]{No3/fig/approach-img.png}
\caption{サンプルイメージ図}
\label{fig:approach-img.png}
\end{figure}
\end{minipage}
\begin{minipage}{0.5\textwidth}
\begin{figure}[H]
\centering
\includegraphics{No3/fig/approach-crop.pdf}
\caption{アプローチ線}
\label{fig:approach.pdf}
\end{figure}
\end{minipage}

\begin{figure}[H]
\centering
\includegraphics[scale=0.55]{No3/fig/sample3-1.png}
\caption{アプローチ線を追加したGコードシミュレーション画面}
\label{fig:sample3-1.png}
\end{figure}

\subsubsection{強制移動指示レイヤ}
 図\ref{fig:sample3-1.png} のシミュレーション結果を見ても解る通り,原点へ戻るときもワークと干渉してしまう可能性があります.
加工開始位置指示レイヤは,最初の加工位置やアプローチを指示するものでしたが,
強制移動指示レイヤは,次の切削データの検索途中でNCVCに移動を指示する情報となります.
つまり,次の切削領域への移動,Z軸の上下が必要なときに強制移動指示レイヤが参照されます.

\begin{minipage}[t]{0.5\textwidth}
 強制移動指示レイヤは線データのみを認識します.
図\ref{fig:ReadSetup3.png} で設定した移動レイヤに作図して下さい.
また,必ず切削データと接続されていなければなりません.
図\ref{fig:move.pdf} は切削が終わったあとの移動軌跡を作図したものです.
\end{minipage}
\begin{minipage}[t]{0.5\textwidth}
\vspace*{-2zh}
\begin{figure}[H]
\centering
\includegraphics{No3/fig/move-crop.pdf}
\caption{強制移動指示レイヤを追加}
\label{fig:move.pdf}
\end{figure}
\end{minipage}

\begin{figure}[H]
\centering
\includegraphics[scale=0.55]{No3/fig/sample3-2.png}
\caption{強制移動データを追加したGコードシミュレーション画面}
\label{fig:sample3-2.png}
\end{figure}

\begin{minipage}[t]{0.5\textwidth}
 図\ref{fig:sample3-2.png} のシミュレーション結果から,強制移動指示レイヤがR点で移動していることが解ります.
この設定は加工条件の[レイヤ]タブにあります.
今回の例では[イニシャル点復帰]が正解ですが,強制移動指示レイヤは大抵の場合,次の切削領域への移動制御に使われるため,
通常は[R点復帰]で問題無いと思います.
\end{minipage}
\begin{minipage}[t]{0.5\textwidth}
\vspace*{-2zh}
\begin{figure}[H]
\centering
\includegraphics[scale=0.7]{No3/fig/move-setup.png}
\caption{強制移動指示レイヤのZ値設定}
\label{fig:move-setup.png}
\end{figure}
\end{minipage}

\begin{itembox}[l]{ここまでの【まとめ】}
(1) 加工開始位置指示レイヤ
\begin{itemize}
\item 加工開始位置を円で示す,またはアプローチ線を作図
\item 円や線は切削データと繋がって無くても良い
\end{itemize}
(2) 強制移動指示レイヤ
\begin{itemize}
\item 切削データが途切れ,次の切削領域へ移動するとき,参照される
\item 切削データと繋がっていなければならない
\item 強制移動指示レイヤのZ値は加工条件の設定による
\end{itemize}
\end{itembox}

\subsection{Gコード(文字)の埋め込み}
 NCVCで生成されるGコードは,基本的に位置決めと直線・円弧補間のG00~G03,固定サイクルG81~の一部だけです.
ここでは工作機械の特殊コード(例えばATCによるツール交換コード)やカスタムマクロ呼び出しコードなど,
生成されるGコードに任意の文字列を埋め込む方法を解説します.

\begin{minipage}[t]{0.5\textwidth}
 作図方法は至極簡単.図\ref{fig:moji.pdf} に示すように,埋め込みたいオブジェクト(線や円弧)の端点に文字を作図するだけです.
文字データは原点レイヤ以外,すなわち,切削レイヤと2つの移動レイヤ,
それから【\ref{sec:move} 移動レイヤ】の節で解説した最後のレイヤである[コメント文字列挿入用レイヤ](p.\pageref{fig:ReadSetup3.png} 図\ref{fig:ReadSetup3.png})に作図することができます.

 埋め込みタイミングは,文字専用レイヤである[コメント文字列挿入用レイヤ]が先に参照され,次に各レイヤタイプに属する文字データが参照されます.
[コメント文字列挿入用レイヤ]に書かれた文字はその名の通りGコードに対するコメントと見なされ,カッコで括られます.その他のレイヤに書かれた文字はそのまま出力されます.
なお,文字データは1行で作図する必要がありますが,複数行の任意Gコードを埋め込みたい場合は改行位置で ``\,\textbackslash n\,'' と入力して下さい.
\end{minipage}
\begin{minipage}[t]{0.5\textwidth}
\vspace*{-1zh}
\begin{figure}[H]
\centering
\includegraphics{No3/fig/moji-crop.pdf}
\caption{文字埋め込みサンプル}
\label{fig:moji.pdf}
\end{figure}
\end{minipage}

 図\ref{fig:moji.pdf} の作図から生成されるGコードは以下の通りです.
反転部分が文字情報から拾われたデータです.
作図は簡単ですが,文字を埋め込むレイヤと場所(タイミング)には若干知恵を絞る必要があります.

\begin{minipage}[t]{0.5\textwidth}
\begin{screen}
{\small\texttt{\%\\[-0.5zh]
G90G54G92X0Y0Z10.\\[-0.5zh]
M8\\[-0.5zh]
M68\\[-0.5zh]
S3000M3\\[-0.5zh]
\colorbox{black}{\textcolor{white}{T02}}\\[-0.5zh]
G00X10.Y25.\\[-0.5zh]
\colorbox{black}{\textcolor{white}{G65P1000A10}}\\[-0.5zh]
X25.Y0\\[-0.5zh]
\colorbox{black}{\textcolor{white}{G91G28Z0}}\\[-0.5zh]
\colorbox{black}{\textcolor{white}{M06}}\\[-0.5zh]
\colorbox{black}{\textcolor{white}{G90G54T03}}\\[-0.5zh]
Z1.\\[-0.5zh]
G01Z-2.F100\\
 右へ続く$\nearrow$
}}
\end{screen}
\end{minipage}
\begin{minipage}[t]{0.5\textwidth}
\begin{screen}
{\small\texttt{Y50.F300\\[-0.5zh]
\colorbox{black}{\textcolor{white}{G91G28Z0}}\\[-0.5zh]
\colorbox{black}{\textcolor{white}{M06}}\\[-0.5zh]
\colorbox{black}{\textcolor{white}{G90G54T01}}\\[-0.5zh]
G00Z1.\\[-0.5zh]
X40.Y25.\\[-0.5zh]
\colorbox{black}{\textcolor{white}{G65P1001A10}}\\[-0.5zh]
Z10.\\[-0.5zh]
M9\\[-0.5zh]
M5\\[-0.5zh]
X0Y0\\[-0.5zh]
M30\\[-0.5zh]
\%
}}
\end{screen}
\end{minipage}

\begin{figure}[H]
\centering
\includegraphics[scale=0.55]{No3/fig/sample4.png}
\caption{Gコードシミュレーション画面}
\label{fig:sample4.png}
\end{figure}

\subsection{深彫切削(Z軸ステップ切削)}
 ここでは[深彫切削]を解説します.
細い刃物での切削で1回の切り込み量に制限がある場合等,Z軸を段階的に下げて切削するコードを生成することができます.

\begin{minipage}[t]{0.5\textwidth}
 深彫切削は加工条件で指示します.
サンプルデータは【\ref{DesignCAD} CADでの作図】を使いましょう.
同じく【\ref{sec:init.nci} 加工条件の設定】の要領で条件ファイルを開き,
[深彫]のタブをクリックして下さい(図\ref{fig:deep.png}).

 ポイントは2点.
まず[基本切り込み]の値ですが,ここでは ``\,1回目の切り込み量\,'' と解釈して下さい.
これは基本タブから参照されている値です.
このタブで値の変更はできません.

 次に深彫切削グループ一連の設定.
[最終切り込み]で最終的に必要な深さ,[切り込みステップ]にて刃物が一度に切り込む量を指定します.
とりあえずこの設定情報でGコードを生成してみると,図\ref{fig:deep.png} のようになります.

 その他の加工条件で代表的なもの[手順]は連続線.
グループを先に彫り進める(一筆)か,同一Z値で全体を彫り進める(全体)かの設定.
[方向]は次のZ値を切削するときにそのまま戻る(往復)か,一旦最初の加工座標まで戻る(一方)かの設定です.
いずれもシミュレーション画面で明確に出ますので,状況に応じて設定して下さい.
残りの設定は【リファレンス】で解説します.
\end{minipage}
\begin{minipage}[t]{0.5\textwidth}
\vspace*{-2zh}
\begin{figure}[H]
\centering
\includegraphics[scale=0.7]{No3/fig/deep-setup.png}
\caption{深彫の設定}
\label{fig:deep-setup.png}
\end{figure}
\end{minipage}

\begin{figure}[H]
\centering
\includegraphics[scale=0.55]{No3/fig/deep.png}
\caption{深彫のシミュレーション結果}
\label{fig:deep.png}
\end{figure}
